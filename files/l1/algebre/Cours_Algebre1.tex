\documentclass[12pt,a4paper]{article}
\usepackage[utf8]{inputenc}
\usepackage[T1]{fontenc}
\usepackage{amsmath, amssymb, amsthm}
\usepackage{geometry}

\geometry{a4paper, margin=1in}

\title{Cours d'Algèbre Linéaire : \\ Espaces Vectoriels, Sous-Espaces Vectoriels et Applications Linéaires}
\author{}
\date{}

\begin{document}

\maketitle

\section*{Introduction}
L'algèbre linéaire est une branche des mathématiques fondamentales, essentielle dans plusieurs domaines, notamment la physique, l'informatique, et l'économie. Ce cours aborde trois concepts clés : les espaces vectoriels, les sous-espaces vectoriels et les applications linéaires.

\section{Espaces Vectoriels}
\subsection*{Définition}
Un \textbf{espace vectoriel} est un ensemble \( V \) muni de deux opérations :
\begin{itemize}
    \item \textbf{Addition} \( + : V \times V \to V \),
    \item \textbf{Multiplication par un scalaire} \( \cdot : \mathbb{K} \times V \to V \),
\end{itemize}
où \( \mathbb{K} \) est un corps (souvent \( \mathbb{R} \) ou \( \mathbb{C} \)).

Ces opérations doivent satisfaire les 8 axiomes suivants :
\begin{enumerate}
    \item \textbf{Associativité de l'addition} : \( u + (v + w) = (u + v) + w \),
    \item \textbf{Commutativité de l'addition} : \( u + v = v + u \),
    \item \textbf{Élément neutre pour l'addition} : Il existe \( 0 \in V \) tel que \( u + 0 = u \),
    \item \textbf{Inverse pour l'addition} : Pour tout \( u \in V \), il existe \( -u \in V \) tel que \( u + (-u) = 0 \),
    \item \textbf{Compatibilité avec la multiplication scalaire} : \( \alpha \cdot (u + v) = \alpha \cdot u + \alpha \cdot v \),
    \item \textbf{Distributivité sur les scalaires} : \( (\alpha + \beta) \cdot u = \alpha \cdot u + \beta \cdot u \),
    \item \textbf{Associativité de la multiplication scalaire} : \( \alpha \cdot (\beta \cdot u) = (\alpha \beta) \cdot u \),
    \item \textbf{Multiplication par l'identité} : \( 1 \cdot u = u \).
\end{enumerate}

\subsection*{Exemples}
\begin{itemize}
    \item \( \mathbb{R}^n \) avec l'addition et la multiplication classique.
    \item L'ensemble des polynômes de degré inférieur ou égal à \( n \), \( \mathbb{P}_n(\mathbb{R}) \).
    \item L'espace des fonctions continues \( C([a, b]) \).
\end{itemize}

\section{Sous-Espaces Vectoriels}
\subsection*{Définition}
Un \textbf{sous-espace vectoriel} \( W \) d'un espace vectoriel \( V \) est un sous-ensemble de \( V \) qui :
\begin{enumerate}
    \item Contient l'élément neutre \( 0 \),
    \item Est fermé pour l'addition : Si \( u, v \in W \), alors \( u + v \in W \),
    \item Est fermé pour la multiplication par un scalaire : Si \( \alpha \in \mathbb{K} \) et \( u \in W \), alors \( \alpha \cdot u \in W \).
\end{enumerate}

\subsection*{Exemples}
\begin{itemize}
    \item \( \{(x, y) \in \mathbb{R}^2 \mid y = 0 \} \) est un sous-espace de \( \mathbb{R}^2 \).
    \item L'ensemble des solutions d'un système linéaire homogène \( A \mathbf{x} = 0 \).
\end{itemize}

\section{Applications Linéaires}
\subsection*{Définition}
Une \textbf{application linéaire} \( f : V \to W \) entre deux espaces vectoriels \( V \) et \( W \) est une fonction qui vérifie :
\begin{enumerate}
    \item \( f(u + v) = f(u) + f(v) \) pour tous \( u, v \in V \),
    \item \( f(\alpha \cdot u) = \alpha \cdot f(u) \) pour tout \( \alpha \in \mathbb{K} \) et \( u \in V \).
\end{enumerate}

\subsection*{Exemples}
\begin{itemize}
    \item \( f : \mathbb{R}^2 \to \mathbb{R}^2 \), définie par \( f(x, y) = (2x, 3y) \).
    \item Dérivation \( D : C^1([a, b]) \to C([a, b]) \), \( D(f) = f' \).
\end{itemize}

\subsection*{Représentation par des Matrices}
Toute application linéaire \( f : \mathbb{R}^n \to \mathbb{R}^m \) peut être représentée par une matrice \( A \in M_{m \times n}(\mathbb{R}) \), telle que pour tout vecteur \( \mathbf{x} \in \mathbb{R}^n \),
\[
f(\mathbf{x}) = A \mathbf{x}.
\]

\subsubsection*{Exemple}
Considérons l'application \( f : \mathbb{R}^2 \to \mathbb{R}^3 \) définie par :
\[
f(x, y) = (x + y, 2x - y, x).
\]
Sa matrice associée dans les bases canoniques est :
\[
A = \begin{bmatrix}
1 & 1 \\
2 & -1 \\
1 & 0
\end{bmatrix}.
\]

\subsection*{Noyau et Image}
\begin{itemize}
    \item \textbf{Noyau} : \( \ker(f) = \{\mathbf{v} \in V \mid f(\mathbf{v}) = 0 \} \), sous-espace de \( V \).
    \item \textbf{Image} : \( \operatorname{Im}(f) = \{f(\mathbf{v}) \mid \mathbf{v} \in V \} \), sous-espace de \( W \).
\end{itemize}

\subsection*{Théorème Fondamental}
Pour toute application linéaire \( f : V \to W \),
\[
\dim(V) = \dim(\ker(f)) + \dim(\operatorname{Im}(f)).
\]

\section*{Exercices}
\begin{enumerate}
    \item Démontrer que l'ensemble \( W = \{(x, y, z) \in \mathbb{R}^3 \mid x + y + z = 0\} \) est un sous-espace de \( \mathbb{R}^3 \).
    \item Trouver le noyau et l'image de l'application linéaire \( f : \mathbb{R}^3 \to \mathbb{R}^2 \) définie par \( f(x, y, z) = (x + y, y + z) \).
    \item Représenter l'application \( f : \mathbb{R}^2 \to \mathbb{R}^3 \), donnée par \( f(x, y) = (x - y, x + y, 2x) \), par une matrice.
    \item Montrer que l'ensemble des matrices symétriques \( S_n(\mathbb{R}) \) est un sous-espace de \( M_n(\mathbb{R}) \).
\end{enumerate}

\section*{Conclusion}
Ce cours a introduit les bases des espaces vectoriels, leurs sous-espaces et les applications linéaires, notamment leur représentation matricielle. Ces concepts sont essentiels pour des sujets avancés comme les transformations linéaires, les valeurs propres et les applications pratiques en sciences.

\end{document}
