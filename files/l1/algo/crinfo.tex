\documentclass{article}
\usepackage[utf8]{inputenc}
\usepackage[T1]{fontenc}
\usepackage{amsmath}
\usepackage{amsfonts}
\usepackage{amssymb}
\usepackage{listings}
\usepackage{xcolor}
\usepackage{array}
\usepackage{longtable}
\usepackage{booktabs} % For better table aesthetics

\lstset{
    basicstyle=\ttfamily\scriptsize,
    keywordstyle=\color{blue},
    commentstyle=\color{gray},
    stringstyle=\color{orange},
    numbers=left,
    numberstyle=\tiny,
    stepnumber=1,
    numbersep=3pt,
    backgroundcolor=\color{white},
    showspaces=false,
    showstringspaces=false,
    showtabs=false,
    frame=lines,
    tabsize=1,
    captionpos=b,
    breaklines=true,
    breakatwhitespace=false,
    escapeinside={\%*}{*)}
}

\title{Initiation à l'informatique et à l'algorithmique}
\author{}
\date{}

\begin{document}

\maketitle

\section*{1. Codage binaire}

\subsection*{Conversion de la base 2 à la base 10}
Le nombre en base 2 est converti en base 10 en utilisant la somme des puissances de 2.

\textbf{Exemple :} \\
Nombre binaire : \( 100101 \) \\
En base 10 : \( 1 \times 2^5 + 0 \times 2^4 + 0 \times 2^3 + 1 \times 2^2 + 0 \times 2^1 + 1 \times 2^0 = 37 \)

\subsection*{Conversion de la base 10 à la base 2}
On divise successivement le nombre par 2, en notant les restes jusqu'à obtenir un quotient nul. Les restes, lus de bas en haut, forment le nombre binaire.

\textbf{Exemple :} \\
Nombre décimal : \( 37 \) \\
En base 2 : \( 100101 \)

\subsection*{Conversion entre bases 2, 8 et 16}
\begin{itemize}
    \item \textbf{Base 2 à Base 8 :} Regrouper les bits par paquets de 3 (en partant de la droite) et convertir chaque paquet en son équivalent décimal.
    \item \textbf{Base 2 à Base 16 :} Regrouper les bits par paquets de 4 et convertir chaque paquet en son équivalent hexadécimal.
    \item \textbf{Base 8 ou 16 à Base 2 :} Convertir chaque chiffre octal (ou hexadécimal) en un groupe de 3 (ou 4) bits.
\end{itemize}

\subsection*{Conversion de la base 10 à la base 8 ou 16}
\begin{itemize}
    \item \textbf{Base 8 :} Diviser successivement par 8, puis lire les restes du bas vers le haut.
    \item \textbf{Base 16 :} Diviser successivement par 16 et utiliser les chiffres A, B, C, D, E, F pour les restes de 10 à 15.
\end{itemize}

\textbf{Exemple :} \\
Nombre décimal : \( 93 \) \\
En base 16 : \( 5D \)

\section*{2. Les variables}
Une variable est une zone de mémoire réservée pour stocker des valeurs. Chaque variable a un type et un nom.


\begin{longtable}{|l|l|l|l|}
\hline
\textbf{Type} & \textbf{Signification} & \textbf{Taille (en octets)} & \textbf{Plage de valeurs} \\
\hline
\endfirsthead

\hline
\textbf{Type} & \textbf{Signification} & \textbf{Taille (en octets)} & \textbf{Plage de valeurs} \\
\hline
\endhead

\hline
\multicolumn{4}{|r|}{\textit{Suite à la page suivante...}} \\
\hline
\endfoot

\hline
\endlastfoot

char & Caractère UTF-16 & 2 octets & 0 à 65535 (Unicode) \\
\hline
byte & Entier signé de 8 bits & 1 octet & -128 à 127 \\
\hline
short & Entier signé de 16 bits & 2 octets & -32 768 à 32 767 \\
\hline
int & Entier signé de 32 bits & 4 octets & ≈ -2,1E9 à 2,1E9 \\
\hline
long & Entier signé de 64 bits & 8 octets & ≈ -9,2E18 à 9,2E18 \\
\hline
float & Nombre à virgule flottante de 32 bits & 4 octets & ±1.4E-45 à ±3.4E38 \\
\hline
double & Nombre à virgule flottante de 64 bits & 8 octets & ±4.9E-324 à ±1.8E308 \\
\hline
boolean & Valeur booléenne (vrai/faux) & 1 bit (en pratique 1 octet) & true / false \\
\hline

\end{longtable}



\section*{3. Les Opérateurs}

\begin{tabular}{|l|l|p{8cm}|}
\hline
\textbf{Catégorie} & \textbf{Opérateur} & \textbf{Description} \\ 
\hline
Concaténation & + & Combine des chaînes de caractères : "Hello" + " World" → "Hello World" \\ 
\hline
Logiques & \&\& & ET logique : vrai si les deux conditions sont vraies \\ 
& || & OU logique : vrai si au moins une condition est vraie \\ 
& ! & NON logique : inverse la valeur de vérité \\ 
\hline
Relationnels & == & Égalité : vrai si les valeurs sont égales \\ 
& != & Différence : vrai si les valeurs sont différentes \\ 
& > & Supérieur à : vrai si la première valeur est supérieure \\ 
& < & Inférieur à : vrai si la première valeur est inférieure \\ 
& >= & Supérieur ou égal à \\ 
& <= & Inférieur ou égal à \\ 
\hline
Arithmétiques & + & Addition \\ 
& - & Soustraction \\ 
& * & Multiplication \\ 
& / & Division \\ 
& \% & Modulo : reste de la division \\ 
\hline
\end{tabular}

\section*{4. Boucles}
Les boucles permettent de répéter une séquence d'instructions un certain nombre de fois ou jusqu'à ce qu'une condition soit remplie.

\begin{lstlisting}[language=Java]
// Exemple de boucle for: utile pour repeter un nombre fixe de fois
System.out.println("Boucle for :");
for (int i = 0; i < 5; i++) {
    // Le code a executer dans chaque iteration
}
System.out.println("Iteration for : " + i);

// Exemple de boucle for imbriquee : utile pour travailler avec des structures multidimensionnelles
System.out.println("\nBoucle for imbriquee :");
for (int i = 1; i <= 3; i++) {
    for (int j = 1; j <= 2; j++) {
        // Le code a executer dans chaque iteration de la boucle imbriquee
    }
    System.out.println("i : " + i + " j : " + j);
}
\end{lstlisting}

\section*{5. Les conditions}
Les conditions permettent d'exécuter une portion de code seulement si une condition donnée est remplie (\texttt{if}, \texttt{else}, \texttt{switch}).
\begin{lstlisting}[language=Java]
// Exemple d'utilisation de if, else if, et else
int age = 20;
// Verifie si l'age est superieur ou egal a 18
if (age >= 18) {
    System.out.println("Vous etes adulte.");
}
// Si age est entre 13 et 17, on considere l'utilisateur comme adolescent
else if (age >= 13) {
    System.out.println("Vous etes adolescent.");
}
// Si aucune condition precedente n'est remplie, on affiche "enfant"
else {
    System.out.println("Vous etes enfant.");
}

// Exemple de boucle while
int compteur = 0;
// Tant que compteur est inferieur a 5, la boucle continue
while (compteur < 5) {
    System.out.println("Compteur (while) : " + compteur);
    compteur++; // Incremente compteur de 1 a chaque iteration
}

// Reinitialise le compteur pour la boucle do-while
compteur = 0;
// Exemple de boucle do-while
do {
    System.out.println("Compteur (do-while) : " + compteur);
    compteur++; // Incremente compteur de 1 a chaque iteration
} while (compteur < 5); // Verifie la condition apres l'execution du bloc
\end{lstlisting}

\section*{6. Les chaînes de caractères (String)}
Une chaîne de caractères est une suite de caractères encadrée par des guillemets doubles. Exemple :
\begin{lstlisting}[language=Java]
// Creation d'un objet Scanner pour lire la saisie clavier
Scanner scanner = new Scanner(System.in);
// Demande a l'utilisateur de saisir par clavier
String S = scanner.nextLine();
// Declaration et initialisation d'une chaine de caracteres
String originalString = "Bonjour, monde!";

// 1. Utilisation de length() pour obtenir la longueur de la chaine
int length = originalString.length();
System.out.println("Longueur de la chaine: " + length); // Affiche la longueur de la chaine

// 2. Utilisation de charAt() pour obtenir un caractere a un index donne
char firstChar = originalString.charAt(0); // Le premier caractere
char lastChar = originalString.charAt(length - 1); // Le dernier caractere
System.out.println("Premier caractere: " + firstChar); // Affiche 'B'
System.out.println("Dernier caractere: " + lastChar); // Affiche '!'

// 3. Utilisation de substring() pour obtenir une sous-chaine
String subString = originalString.substring(0, 7); // Obtient "Bonjour"
System.out.println("Sous-chaine: " + subString);

// 4. Utilisation de indexOf() pour trouver l'index d'un caractere
int indexOfO = originalString.indexOf('o'); // Trouve le premier 'o'
System.out.println("Index du premier 'o': " + indexOfO); // Affiche 1

// 5. Utilisation de equals() pour comparer des chaines
String anotherString = "Bonjour, monde!";
boolean areEqual = originalString.equals(anotherString); // Compare les deux chaines
System.out.println("Les chaines sont egales: " + areEqual); // Affiche true
\end{lstlisting}


\section*{7. Les Tableaux}
La déclaration d'un tableau prend cette forme :

<type>[] <nomDuTableau> = {<valeur1>, <valeur2>, ..., <valeurN>};

\begin{lstlisting}[language=Java]
public class ArrayExample {
    public static void main(String[] args) {
        // 1. Declaration et initialisation d'un tableau unidimensionnel
        int[] oneDimensionalArray = {1, 2, 3, 4, 5}; // Initialisation correcte
        // Afficher les elements du tableau unidimensionnel
        System.out.println("Tableau unidimensionnel :");
        for (int i = 0; i < oneDimensionalArray.length; i++) {
            System.out.print(oneDimensionalArray[i] + " "); // Affiche chaque element
        }
        System.out.println(); // Nouvelle ligne apres l'affichage du tableau

        // 2. Declaration et initialisation d'un tableau bidimensionnel (matrice)
        int[][] twoDimensionalArray = {
            {1, 2, 3},
            {4, 5, 6},
            {7, 8, 9}
        };
        // Afficher les elements du tableau bidimensionnel
        System.out.println("\nTableau bidimensionnel :");
        for (int i = 0; i < twoDimensionalArray.length; i++) {
            for (int j = 0; j < twoDimensionalArray[i].length; j++) {
                System.out.print(twoDimensionalArray[i][j] + " "); // Affiche chaque element
            }
            System.out.println(); // Nouvelle ligne apres chaque ligne de la matrice
        }

        // 3. Modification d'un element dans le tableau unidimensionnel
        oneDimensionalArray[2] = 10; // Modifie l'element a l'index 2 (le troisieme element)
        System.out.println("\nTableau unidimensionnel apres modification :");
        for (int num : oneDimensionalArray) {
            System.out.print(num + " "); // Affiche les elements du tableau modifie
        }

        // 4. Calcul de la somme des elements dans le tableau bidimensionnel
        int sum = 0;
        for (int i = 0; i < twoDimensionalArray.length; i++) {
            for (int j = 0; j < twoDimensionalArray[i].length; j++) {
                sum += twoDimensionalArray[i][j]; // Ajoute chaque element a la somme
            }
        }
        System.out.println("\n\nSomme des elements du tableau bidimensionnel: " + sum);

        // 5. Recherche d'un element dans le tableau unidimensionnel
        int searchElement = 10; // element e rechercher
        boolean found = false;
        for (int i = 0; i < oneDimensionalArray.length; i++) { // Recherche dans le tableau
            if (oneDimensionalArray[i] == searchElement) { // Si trouve
                found = true;
                System.out.println("element " + searchElement + " trouve a l'index " + i);
                break; // Sort de la boucle si l'element est trouve
            }
        }
        if (!found) {
            System.out.println("element " + searchElement + " non trouve dans le tableau.");
        }
    }
}
\end{lstlisting}




\section*{8. Les fichiers texte}
Java permet la lecture et l'écriture de fichiers texte en utilisant les classes \texttt{BufferedReader}, \texttt{BufferedWriter}, etc.
\begin{lstlisting}[language=Java]
// Exemple de lecture et ecriture dans un fichier
import java.io.*;
import java.util.Scanner;

public class FileReadWriteExample {
    public static void main(String[] args) throws IOException {
        // Nom du fichier dans lequel ecrire et lire
        String fileName = "example.txt";

        // 1. Ecriture dans le fichier avec BufferedWriter
        BufferedWriter writer = new BufferedWriter(new FileWriter(fileName));
        writer.write("Bonjour, ceci est un exemple de fichier texte.");
        writer.newLine(); // Insere une nouvelle ligne
        writer.write("Voici une autre ligne.");
        writer.newLine(); // Insere une autre ligne
        writer.write("Derniere ligne de cet exemple.");
        writer.close(); // Ferme le writer manuellement
        System.out.println("ecriture terminee dans le fichier : " + fileName);

        // 2. Lecture et recherche dans le fichier avec BufferedReader
        System.out.println("\n--- Lecture du fichier et recherche de contenu ---");

        // Demande a l'utilisateur d'entrer une chaine a rechercher
        Scanner scanner = new Scanner(System.in);
        System.out.print("Entrez le texte a rechercher dans le fichier : ");
        String searchString = scanner.nextLine();
        scanner.close(); // Ferme le scanner manuellement

        // Lecture du fichier ligne par ligne
        BufferedReader reader = new BufferedReader(new FileReader(fileName));
        String ligneLue;
        boolean found = false;

        while ((ligneLue = reader.readLine()) != null) {
            if (ligneLue.contains(searchString)) {
                System.out.println("Chaine trouvee : " + ligneLue);
                found = true;
            }
        }
        reader.close(); // Ferme le reader manuellement

        if (!found) {
            System.out.println("La chaine \"" + searchString + "\" n'a pas ete trouvee dans le fichier.");
        }
    }
}
\end{lstlisting}
\section*{9. Maths en Java}
\begin{tabular}{|l|l|}
\hline
\textbf{Méthode} & \textbf{Description} \\ 
\hline
Math.abs(x) & Retourne la valeur absolue \\ 
Math.sqrt(x) & Retourne la racine carrée \\ 
Math.max(x, y) & Retourne le plus grand des deux nombres \\ 
Math.round(x) & Arrondit x à l'entier le plus proche \\ 
Math.min(x, y) & Retourne le plus petit des deux nombres \\ 
Math.random() & Retourne un nombre aléatoire entre 0.0 et 1.0 \\ 
Math.pow(x, y) & Retourne x élevé à la puissance y \\ 
Math.exp(x) & Retourne e\^x \\ 
\hline
\end{tabular}

\section*{10. Actions et fonctions}
En Java, on a les actions et les fonctions. Pour faire une tâche, parfois c'est très compliqué, donc on divise le travail en des petits sous-partis rendant le programme plus structuré. Et là, c'est plus facile à comprendre, à maintenir, et à déboguer

\begin{itemize}
    \item \textbf{Actions (méthodes void)} : Elles sont définies avec le mot-clé void, indiquant qu'elles ne retournent pas de valeur.
    \begin{lstlisting}[language=Java]
// Classe principale
public class PairImpair {
    public static void main(String[] args) {
        verifierPairImpair(7);  // Appelle la methode avec le nombre 7
        verifierPairImpair(10); // Appelle la methode avec le nombre 10
    }

    // Methode void (action) qui affiche si un nombre est pair ou impair
    public static void verifierPairImpair(int nombre) {
        System.out.println(nombre + " est " + (nombre % 2 == 0 ? "pair" : "impair") + ".");
    }
}
\end{lstlisting}
    \item \textbf{Fonctions (méthodes avec retour)} : retournent une valeur d'un type spécifique (\texttt{int}, \texttt{double}, \texttt{String}, etc.)En plus la méthode doit inclure une instruction return pour fournir le résultat.
\end{itemize}
 \begin{lstlisting}[language=Java]
// Methode principale
public static void main(String[] args) {
    int nombre = 7; // Exemple de nombre a tester
    if (estPair(nombre)) {
        System.out.println(nombre + " est pair.");
    } else {
        System.out.println(nombre + " est impair.");
    }
}

// Fonction qui retourne true si le nombre est pair, sinon false
public static boolean estPair(int nombre) {
    return nombre % 2 == 0;
}
\end{lstlisting}

\end{document}