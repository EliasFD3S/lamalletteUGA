
\documentclass[a4paper,10pt]{article}
\usepackage{amsmath}
\usepackage{amsfonts}
\usepackage{amssymb}
\usepackage{geometry}
\usepackage{array}
\usepackage{enumitem}
\usepackage{graphicx}
\geometry{top=1in, bottom=1in, left=1in, right=1in}

\usepackage[utf8]{inputenc}



\title{Synthèse Analyse 1}
\date{}

\begin{document}
\maketitle

\section*{Logique Mathématique}
\subsection*{Quantificateurs et Connecteurs Logiques}

\begin{center}
\footnotesize % Réduction de la taille de la police
\begin{tabular}{|c|c|c|}
\hline
\textbf{Notations} & \textbf{Type} & \textbf{Définition} \\
\hline
\( \forall x \in E \) & Quantificateur Universel & Pour tout \( x \) dans \( E \)\\
\hline
\( \exists x \in E \) & Quantificateur Existentiel & Il existe au moins un \( x \) dans \( E \)\\
\hline
\( \exists! x \in E \) & Quantificateur Unique & Il existe un unique \( x \) dans \( E \)\\
\hline
\( \neg P \) ou \textbf{non \(P\)} & Négation & L'assertion est vraie si \( P \) est fausse. \\
\hline
\( P \cap Q \) & Conjonction & Vraie seulement si \( P \) et \( Q \) sont toutes deux vraies. \\
\hline
\( P \cup Q \) & Disjonction & Vraie si au moins une des assertions \( P \) ou \( Q \) est vraie. \\
\hline
\( P \implies Q \) & Implication & Faux seulement si \( P \) est vrai et \( Q \) est faux. Vrai sinon \\
\hline
\( P \iff Q \) & Équivalence & Vraie si \( P \) et \( Q \) sont toutes deux vraies ou toutes deux fausses. \\
\hline
\end{tabular}
\end{center}
\begin{table}[h!]
\centering
\renewcommand{\arraystretch}{1.5} % Augmente l'espacement vertical des lignes
\begin{tabular}{|c|c|c|}
\hline
\textbf{Quantificateurs} & \textbf{Négation} & \textbf{Stratégie} \\
\hline
$\forall x, p(x)$ & $\exists x, \overline{p(x)}$ & Trouver un contre-exemple \\
\hline
$\exists x, p(x)$ & $\forall x, \overline{p(x)}$ & Tout $x$ contredit $p(x)$ \\
\hline
\end{tabular}

\vspace{1em} % Espace vertical entre les deux tableaux

\begin{tabular}{|c|c|c|}
\hline
\textbf{Connecteurs} & \textbf{Négation} & \textbf{Stratégie} \\
\hline
$p(x) \text{ et } q(x)$ & $\overline{p(x)} \text{ ou } \overline{q(x)}$ & Contredire l’un des prédicats \\
\hline
$p(x) \text{ ou } q(x)$ & $\overline{p(x)} \text{ et } \overline{q(x)}$ & Contredire les deux prédicats \\
\hline
$p(x) \implies q(x)$ & $p(x) \text{ et } \overline{q(x)}$ & Trouver un élément vérifiant $p(x)$ et contredisant $q(x)$ \\
\hline
\end{tabular}
\end{table}



\section*{Bornes Inf/Sup}

\subsection*{Définitions}

\begin{itemize}[label={\tiny$\bullet$}]
	\item  \( a \in E \)  \textbf{maximum} (resp. minimum) de \( A \) si :
	\[
	a \in A \quad \text{et} \quad \forall b \in A, a \geq b \quad (\text{resp. } a \leq b).
	\]
	\item \( M \)  \textbf{majorant} (resp. minorant) de \( A \) si \( \forall b \in A, M \geq b \quad (\text{resp. } m \leq b) \).
	\item Une partie \( A \) est \textbf{bornée} si elle est majorée et minorée.
	\item \textbf{Borne supérieure} : \(x = \sup A\) ssi :
	\[
	\begin{cases}
    	\forall a \in A, \; a \leq x, & \text{(x est un majorant)} \\
    	\forall \varepsilon > 0, \; \exists a \in A \; \text{tel que} \; a > x - \varepsilon. & \text{(et c'est le plus petit)}
	\end{cases}
	\]

	\item \textbf{Borne inférieure } :  \(x = \inf A\) ssi :
	\[
	\begin{cases}
    	\forall a \in A, \; a \geq x, & \text{(x est un minorant)} \\
    	\forall \varepsilon > 0, \; \exists a \in A \; \text{tel que} \; a < x + \varepsilon. & \text{(et c'est le plus grand)}
	\end{cases}
	\]

\end{itemize}


\section*{Suites Numériques}


\begin{itemize}
	\item Suite numérique = ensemble de nombres indexé par les entiers, notée \( (u_n)_{n \in \mathbb{N}} \).
	\item Peut être définie :
	\begin{itemize}
    	\item explicitement : \( u_n = f(n) \),
    	\item Par itération : \( u_{n+1} = f(u_n) \) avec \( u_0 \)

	\end{itemize}
\end{itemize}


\subsection*{Monotonie}
- \( (u_n) \) \textbf{croissante} si \( u_{n+1} \geq u_n \)
-  \textbf{décroissante} si \( u_{n+1} \leq u_n \) (strictement : inégalités strictes) \\
- Etudier \( u_{n+1} - u_n \) et comparer à 0 ou si \( u_{n} \) positive : étudier \( \frac{u_{n+1}}{u_n} \) et comparer à 1.

\subsection*{Bornes}
\begin{itemize}
	\item \((u_n)\) \textbf{majorée} si \(\exists M \in \mathbb{R}, \quad u_n \leq M, \quad \forall n \in \mathbb{N}\)
    
	\item \textbf{minorée} si \(\exists m \in \mathbb{R}, \quad u_n \geq m, \quad \forall n \in \mathbb{N}\)
    
	\item \textbf{bornée} si majorée et minorée.
\end{itemize}

- Les propriétés de bornes s'appliquent à partir d'un certain rang. (on peut étudier pour n grand)

\subsection*{Récurrence}
Pour une propriété \( P_n \) :\\ \textbf{Initialisation} : On montre que \( P_0 \) est vraie. \\
\textbf{Hérédité} : On se place à un certain n tel que \( P_n \) soit vraie. On montre que \( P_n \implies P_{n+1} \) \\\textbf{Conclusion} : \( P_n \) est vraie \(\forall n \in \mathbb{N}\).



\section*{Suites Fondamentales}

\subsection*{Suite Arithmétique}
- Formule explicite (raison r) : \( u_n = u_0 + nr \) \\
- Somme des \( k + 1 \) premiers termes :
  \[
  S_k = \frac{(k+1)(u_0 + u_k)}{2}.
  \]

\subsection*{Suite Géométrique}
- Formule explicite (raison q) : \( u_n = u_0 q^n \) \\
- Somme des \( k + 1 \) premiers termes :
  \[
  S_k = u_0 \frac{1 - q^{k+1}}{1 - q}.
  \]

\subsection*{Suite Arithmético-Géométrique}
Définie par \( u_{n+1} = au_n + b \).
- Si \( a \neq 1 \), le terme général peut être calculé en se ramenant à une suite géométrique.


\section*{Application à la finance}


En finance, , la valeur d'un capital évolue dans le temps, ce qui nécessite une compréhension des notions de \textbf{capitalisation} et \textbf{actualisation} pour effectuer des placements ou des emprunts.

Un taux d'intérêt, souvent exprimé annuellement, peut être converti pour d'autres périodes. Par exemple, un taux annuel de 5\% correspond à un taux mensuel calculé par :
\[
\left( 1 + \frac{5}{100} \right) = \left( 1 + \frac{t}{100} \right)^{12},
\]
où \(t\) est le taux mensuel. En résolvant, on trouve :
\[
t = 0.407\%.
\]

\subsection*{Vocabulaire}
\begin{itemize}
    \item \textbf{Valeur actuelle} : capital initial placé.
    \item \textbf{Valeur future} : capital obtenu après \(n\) périodes.
    \item \textbf{Capitalisation} : calcul de la valeur future.
    \item \textbf{Actualisation} : calcul de la valeur actuelle.
\end{itemize}

\subsection*{Placements à intérêts simples}
Un capital \(C\) produit des intérêts constants \(t\%\) à chaque période. 

Le capital \(C_n\) au bout de \(n\) périodes est donné par :
\[
C_n = C + C \cdot \frac{t}{100} \cdot n.
\]

\subsection*{Placements à intérêts composés}
Les intérêts sont réinvestis, générant des intérêts supplémentaires.

Le capital \(K_n\) après \(n\) périodes est donné par :
\[
K_n = K \left( 1 + \frac{t}{100} \right)^n.
\]

\subsection*{Valeur acquise d'une suite d'annuités}
En plaçant des annuités constantes \(A\) chaque période avec un taux composé \(t\%\), la valeur acquise après \(n\) périodes est :
\[
V_n = A \left[ 1 + \left( 1 + \frac{t}{100} \right) + \cdots + \left( 1 + \frac{t}{100} \right)^n \right].
\]
C'est une somme géométrique :
\[
V_n = A \cdot \frac{1 - \left( 1 + \frac{t}{100} \right)^{n+1}}{1 - \left( 1 + \frac{t}{100} \right)} = A \cdot \frac{1 - \left( 1 + \frac{t}{100} \right)^{n+1}}{\frac{t}{100}}.
\]

\subsection*{Calcul de la mensualité d'un crédit}
La mensualité constante \(m\) pour rembourser un capital \(C\) emprunté sur \(n\) périodes avec un taux \(t\) (décimal) est donnée par :
\[
m = C \cdot \frac{t}{1 - (1 + t)^{-n}}.
\]

\subsection*{Problème de dépréciation}
Considérons un produit initialement en quantité \(Q_i\) qui se déprécie de \(p\%\) par période, tout en ajoutant une production constante \(Q\) chaque période. La quantité finale après \(n\) périodes est :
\[
Q_i \cdot (1 - p)^n + Q \cdot \frac{1 - (1 - p)^n}{p}.
\]
En fixant une quantité cible, cette équation permet de déterminer \(Q\).


\section*{Limites de suites}

Déf : Partie entière $E(\cdot)$ donne pour chaque réel le plus grand entier inférieur ou le plus petit entier supérieur.


\subsubsection*{Limite d'une Suite Numérique}

$(u_n)$ admet une limite $l \in \mathbb{R}$ si :
\[
\forall \varepsilon > 0, \exists N \in \mathbb{N}, \forall n \geq N, |u_n - l| \leq \varepsilon.
\]
$(u_n)$ \textbf{convergente} si admet une limite réelle $l$, (sinon : \textbf{divergente}).


\[
(u_n) \text{ admet une limite } l \in \mathbb{R} \implies l \text{ est unique.}
\]


- Une suite $(u_n)$ tend vers $+\infty$ si :
\[
\forall M \in \mathbb{R}, \exists N \in \mathbb{N}, \forall n \geq N, u_n \geq M.
\]

\section*{Suites de Référence}

- \( (-1)^n \) : alternée \(\implies\) pas de limite. \\
- Suites arithmétiques ayant pour représentation graphique des points alignés : divergentes. \\

- Suite géométrique \( q^n \) :
\[
|q| < 1 \implies \lim_{n \to \infty} q^n = 0,
\]
\[
q = 1 \implies \lim_{n \to \infty} q^n = 1,
\]
\[
q > 1 \implies \lim_{n \to \infty} q^n = +\infty,
\]
\[
q \leq -1 \implies q^n \text{ n'admet pas de limite.}
\]

Limites de référence :
- Constante : \( \lim_{n \to \infty} c = c \), pour \( c \in \mathbb{R} \) ;
- Monômes : \( \lim_{n \to \infty} n^a = +\infty \) pour \( a > 0 \), et \( \lim_{n \to \infty} n^{1/2} = +\infty \) ;
- Par inversion, si \( a < 0 \), \( \lim_{n \to \infty} n^a = 0 \).

\subsection*{Opérations sur les Limites}

NB : "FI" = Forme indéterminée

\begin{enumerate}
	\item \( (u_n) \to l \) et \( (v_n) \to l' \), alors \( (u_n + v_n) \to l + l' \)
	\item Si \( (u_n) \to l \) et \( (v_n) \to +\infty \), alors \( (u_n + v_n) \to +\infty \). Si \( (u_n) \to +\infty \) et \( (v_n) \to -\infty \), on obtient une FI
	\item Si \( (u_n) \to l \) et \( (v_n) \to l' \), alors \( (u_n \times v_n) \to l \times l' \)
	\item Si \( (u_n) \to l \neq 0 \) ou \( (u_n) \to \pm\infty \), et \( (v_n) \to \pm\infty \), alors \( (u_n \times v_n) \to \pm\infty \), selon le signe du produit. Une FI apparaît pour \( 0 \times \pm\infty \)
\end{enumerate}

\subsubsection*{Limite de l'inverse}

Si \( (u_n) \to l \neq 0 \), alors \( \left(\frac{1}{u_n}\right) \to \frac{1}{l} \)

\vspace{0.5cm}

\textbf{Théorème des gendarmes}

Soient \( (u_n) \) et \( (w_n) \) deux suites convergeant vers \( l \in \mathbb{R} \). Si \( (v_n) \) est telle que \( u_n \leq v_n \leq w_n \) à partir d'un certain rang, alors \( (v_n) \) converge vers \( l \), c'est-à-dire \( \lim v_n = l \).

\vspace{0.5cm}

\textbf{Inégalités de suites}

\begin{enumerate}
	\item Si \( u_n \leq v_n \) à partir d'un certain rang et \( \lim u_n = +\infty \), alors \( \lim v_n = +\infty \).
	\item Si \( u_n \geq v_n \) à partir d'un certain rang et \( \lim u_n = -\infty \), alors \( \lim v_n = -\infty \).
\end{enumerate}

\vspace{0.5cm}

\textbf{Suites croissantes et décroissantes}

\begin{enumerate}
	\item Si \( (u_n) \) est croissante et majorée, alors \( (u_n) \) est convergente et \( \lim u_n = \sup \{ u_n : n \in \mathbb{N} \} \).
	\item Si \( (u_n) \) est décroissante et minorée, alors \( (u_n) \) est convergente et \( \lim u_n = \inf \{ u_n : n \in \mathbb{N} \} \).
\end{enumerate}

\vspace{0.5cm}

\textbf{Suites adjacentes}

\( (u_n) \) et \( (v_n) \) sont dites adjacentes si :
\[
- u_n \leq v_n \quad \forall n \in \mathbb{N}, \quad (u_n) \text{ croissante }, \quad (v_n) \text{ décroissante }, \quad \lim_{n \to \infty} (v_n - u_n) = 0.
\]

Si \( (u_n) \) et \( (v_n) \) sont adjacentes, alors elles convergent toutes les deux vers \( l \in \mathbb{R} \), avec \( u_n \leq l \leq v_n \) pour tout \( n \in \mathbb{N} \).

\subsection*{Suites définies par une relation de récurrence : \( u_{n+1} = f(u_n) \)}
Soit \( (u_n)_{n \geq 0} \) une suite définie par récurrence, avec \( u_0 \) donné et \( u_{n+1} = f(u_n) \). Si \( (u_n) \) converge vers une limite \( l \), et si \( f \) est continue, alors :
\[
f(l) = l,
\]

\textbf{Point fixe : }
\( x \in I \) \emph{point fixe} de \( f \) si \( f(x) = x \).

\subsection*{Convergence vers un Point Fixe}
Pour une suite définie par \( u_{n+1} = f(u_n) \), les conditions suivantes permettent de garantir que \( \lim u_n = l \) :
\begin{itemize}
	\item Si \( f(I) \subset I \),
	\item \( f \) possède un unique point fixe \( l \) sur \( I \),
	\item \( u_0 \in I \) et \( (u_n) \) est croissante majorée ou décroissante minorée.
\end{itemize}

\textbf{Fonction contractante} \\
Soit \( I \) un intervalle fermé et borné de \( \mathbb{R} \) et \( f : I \to I \). La fonction \( f \) est \emph{contractante} sur \( I \) s'il existe une constante \( k \in [0, 1[ \) telle que :
\[
\forall x, y \in I, \quad |f(x) - f(y)| \leq k |x - y|.
\]

\textbf{Point fixe d'une fonction contractante} \\
Si \( f : I \to I \) est une application contractante, alors :
\begin{itemize}
	\item \( f \) est continue,
	\item \( f \) possède un unique point fixe \( l \in I \),
	\item pour tout \( u_0 \in I \), la suite \( (u_n) \) définie par \( u_0 \) et \( u_{n+1} = f(u_n) \) converge vers \( l \).
\end{itemize}

\end{document}
