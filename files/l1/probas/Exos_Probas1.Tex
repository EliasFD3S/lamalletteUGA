\documentclass[a4paper,12pt]{article}
\usepackage[utf8]{inputenc}
\usepackage[T1]{fontenc}
\usepackage{amsmath, amssymb, amsthm}
\usepackage{geometry}
\usepackage{hyperref}

\geometry{a4paper, margin=1in}

\title{Exercices de Probabilités}
\author{}
\date{}

\begin{document}

\maketitle

\section*{Exercice 1 : Probabilités conditionnelles}

Dans une entreprise, on observe que 70\% des employés sont des hommes et 30\% sont des femmes. Parmi les hommes, 60\% travaillent dans le département A et 40\% dans le département B. Parmi les femmes, 40\% travaillent dans le département A et 60\% dans le département B.

On choisit un employé au hasard. Soit les événements suivants :
\begin{itemize}
    \item $H$ : l’employé est un homme,
    \item $F$ : l’employé est une femme,
    \item $A$ : l’employé travaille dans le département A,
    \item $B$ : l’employé travaille dans le département B.
\end{itemize}

\begin{enumerate}
    \item Quelle est la probabilité qu’un employé choisi au hasard travaille dans le département A ?
    \item On sait que l’employé choisi travaille dans le département A. Quelle est la probabilité que cet employé soit un homme ?
    \item Quelle est la probabilité qu’un employé choisi au hasard travaille dans le département A sachant qu’il est un homme ?
\end{enumerate}

\subsection*{Correction}

\textbf{1. Calcul de la probabilité que l’employé travaille dans le département A}

La probabilité qu’un employé travaille dans le département A est donnée par la loi des probabilités totales :
\[
P(A) = P(A | H)P(H) + P(A | F)P(F),
\]
où $P(H) = 0.70$, $P(F) = 0.30$, $P(A | H) = 0.60$, $P(A | F) = 0.40$.

Ainsi,
\[
P(A) = 0.60 \times 0.70 + 0.40 \times 0.30 = 0.42 + 0.12 = 0.54.
\]

La probabilité qu’un employé choisi au hasard travaille dans le département A est donc $P(A) = 0.54$.

\textbf{2. Probabilité que l’employé soit un homme sachant qu’il travaille dans le département A}

\[
P(H | A) = \frac{P(A | H)P(H)}{P(A)} = \frac{0.60 \times 0.70}{0.54} = \frac{0.42}{0.54} \approx 0.7778.
\]

La probabilité que l’employé soit un homme sachant qu’il travaille dans le département A est environ $0.7778$ ou $77.78\%$.

\textbf{3. Probabilité que l’employé travaille dans le département A sachant qu’il est un homme}

\[
P(A | H) = 0.60.
\]

\section*{Exercice 2 : Fonction de répartition}

Soit $X$ une variable aléatoire discrète ayant la loi suivante :

\[
\begin{array}{c|c|c|c}
X & -1 & 0 & 1 \\
\hline
P(X) & 0.2 & 0.5 & 0.3 \\
\end{array}
\]

Calculez la fonction de répartition $F_X(x)$.

\subsection*{Correction}

La fonction de répartition $F_X(x) = P(X \leq x)$ est calculée comme suit :
\begin{itemize}
    \item $F_X(x) = 0$ pour $x < -1$,
    \item $F_X(x) = 0.2$ pour $-1 \leq x < 0$,
    \item $F_X(x) = 0.2 + 0.5 = 0.7$ pour $0 \leq x < 1$,
    \item $F_X(x) = 0.2 + 0.5 + 0.3 = 1$ pour $x \geq 1$.
\end{itemize}

Ainsi, la fonction de répartition est une fonction en escalier avec les valeurs suivantes :
\[
F_X(-2) = 0, \quad F_X(-1) = 0.2, \quad F_X(0) = 0.7, \quad F_X(1) = 1.
\]

...

\end{document}
