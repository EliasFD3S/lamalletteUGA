\documentclass[a4paper,12pt]{article}
\usepackage[utf8]{inputenc}
\usepackage{amsmath}
\usepackage{graphicx}

\title{Exercices de Statistiques}
\author{}
\date{}

\begin{document}

\maketitle

\section*{Exercice 1 : Analyse univariée des données}
On observe les revenus mensuels (en euros) d’un échantillon de 10 ménages dans une région donnée :

\[ 2200, 2500, 1800, 2100, 2700, 2300, 2600, 2400, 2500, 1900. \]

\begin{enumerate}
    \item Tracer le diagramme en bâtons représentant la distribution des revenus.
    \item Calculer les caractéristiques de position : moyenne, médiane, mode.
    \item Calculer les caractéristiques de dispersion : étendue, variance, écart-type.
    \item Si les revenus augmentent tous de 5 \%, déterminer l’impact sur la moyenne et l’écart-type.
\end{enumerate}

\section*{Exercice 2 : Statistiques bivariées et régression linéaire}
Un responsable d’entreprise souhaite étudier la relation entre le nombre d’heures de formation ($x$) et la productivité ($y$) de ses employés. Les données suivantes sont relevées :

\[
\begin{array}{|c|c|}
\hline
x \text{ (heures)} & 2 & 4 & 6 & 8 & 10 \\
\hline
y \text{ (unités)} & 30 & 50 & 65 & 80 & 95 \\
\hline
\end{array}
\]

\begin{enumerate}
    \item Tracer le nuage de points associé à ces données.
    \item Établir l’équation de la droite de régression par la méthode des moindres carrés.
    \item Calculer le coefficient de corrélation linéaire $r$.
    \item Interpréter le résultat de $r$.
    \item Estimer la productivité pour $x = 7$ heures de formation.
\end{enumerate}

\section*{Exercice 3 : Analyse descriptive des données sur les 100 étudiants}
Une enquête est réalisée sur 100 étudiants concernant leurs habitudes alimentaires. On relève le nombre de repas pris à la cafétéria universitaire dans une semaine. Les données suivantes représentent les effectifs :

\[
\begin{array}{|c|c|c|c|c|c|c|}
\hline
\text{Nombre de repas} & 0 & 1 & 2 & 3 & 4 & 5+ \\
\hline
\text{Effectif} & 10 & 15 & 20 & 25 & 20 & 10 \\
\hline
\end{array}
\]

\begin{enumerate}
    \item Calculer la moyenne, la médiane et le mode du nombre de repas pris.
    \item Tracer le diagramme en bâtons représentant les données.
    \item Calculer l’écart-type.
\end{enumerate}

\section*{Exercice 4 : Régression linéaire simple}
On souhaite modéliser la relation entre la taille ($x$, en cm) et le poids ($y$, en kg) de 10 individus. Les données sont :

\[
\begin{array}{|c|c|c|c|c|c|c|c|c|c|c|}
\hline
x \text{ (cm)} & 160 & 165 & 170 & 175 & 180 & 185 & 190 & 195 & 200 & 205 \\
\hline
y \text{ (kg)} & 55 & 60 & 65 & 70 & 75 & 80 & 85 & 90 & 95 & 100 \\
\hline
\end{array}
\]

\begin{enumerate}
    \item Calculer la droite de régression.
    \item Estimer le poids pour une taille de 178 cm.
    \item Calculer et interpréter le coefficient de corrélation.
\end{enumerate}

\newpage
\section*{Corrections des Exercices}

\subsection*{Correction : Exercice 1}
\begin{enumerate}
    \item Diagramme en bâtons : (Insérer un diagramme créé avec un logiciel graphique ou LaTeX.)
    \item Caractéristiques de position :
    \begin{itemize}
        \item Moyenne : $\bar{x} = 2300 \, \text{euros}$.
        \item Médiane : $Me = 2350 \, \text{euros}$.
        \item Mode : $2500 \, \text{euros}$.
    \end{itemize}
    \item Caractéristiques de dispersion :
    \begin{itemize}
        \item Étendue : $E = 900 \, \text{euros}$.
        \item Variance : $\sigma^2 \approx 108000$.
        \item Écart-type : $\sigma \approx 328.49 \, \text{euros}$.
    \end{itemize}
    \item Impact d’une augmentation de 5 \% :
    \begin{itemize}
        \item Nouvelle moyenne : $2415 \, \text{euros}$.
        \item Nouvel écart-type : $344.91 \, \text{euros}$.
    \end{itemize}
\end{enumerate}

(Continuer de la même manière pour les autres corrections.)

\end{document}
