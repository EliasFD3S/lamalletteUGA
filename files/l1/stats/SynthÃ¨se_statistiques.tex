\documentclass[a4paper,12pt]{article}
\usepackage[utf8]{inputenc}
\usepackage{amsmath}
\usepackage{amsfonts}
\usepackage{amssymb}
\usepackage{graphicx}

\title{Synthèse du Cours de Statistiques}
\author{}
\date{}

\begin{document}

\maketitle

\section*{Statistiques univariées}

\subsection*{Vocabulaire et classification des variables}
\begin{itemize}
    \item \textbf{Variable discrète} : prend un nombre dénombrable de valeurs (exemple : nombre d’élèves ayant une note donnée).
    \item \textbf{Variable continue} : peut prendre toutes les valeurs dans un intervalle (exemple : poids mesuré entre 60 kg et 70 kg).
    \item \textbf{Variables qualitatives et quantitatives} :
    \begin{itemize}
        \item \textit{Qualitative} : regroupe des modalités non numériques (exemple : couleurs).
        \item \textit{Quantitative} : associée à des valeurs numériques.
    \end{itemize}
\end{itemize}

\subsection*{Représentation des données}
\begin{itemize}
    \item \textbf{Diagramme en bâtons} : pour les variables discrètes.
    \item \textbf{Histogramme} : pour les variables continues, où l’aire des rectangles correspond à l’effectif.
    \item \textbf{Fonction de répartition} : permet de répondre à des questions comme "combien d’élèves ont une note inférieure ou égale à 4 ?".
\end{itemize}

\subsection*{Caractéristiques de position}
\begin{itemize}
    \item \textbf{Mode} : valeur la plus fréquente (exemple : 4 est le mode si 3 élèves ont eu cette note).
    \item \textbf{Médiane} : valeur qui sépare la série en deux parties égales. Exemple : si 10 élèves passent un examen, la médiane est la moyenne des notes des 5e et 6e élèves.
    \item \textbf{Moyenne} : somme des valeurs divisée par leur nombre.
    \[
    \bar{x} = \frac{\sum (x_i \cdot n_i)}{N}.
    \]
\end{itemize}

\subsection*{Caractéristiques de dispersion}
\begin{itemize}
    \item \textbf{Étendue} : différence entre la valeur maximale et minimale.
    \item \textbf{Variance et écart-type} :
    \[
    \text{Var}(x) = \frac{\sum (x_i - \bar{x})^2 n_i}{N}.
    \]
    L’écart-type est la racine carrée de la variance ($\sigma$), indiquant l’écart moyen autour de la moyenne.
\end{itemize}

\section*{Statistiques bivariées}

\subsection*{Analyse des relations entre deux variables}
\begin{itemize}
    \item \textbf{Distributions conditionnelles et marginales} :
    \begin{itemize}
        \item \textit{Marginale} : répartition de chaque variable indépendamment (somme sur les lignes ou colonnes).
        \item \textit{Conditionnelle} : répartition d’une variable en fixant une modalité de l’autre.
    \end{itemize}
\end{itemize}

\subsection*{Mesures de dépendance}
\begin{itemize}
    \item Objectif : vérifier si deux variables sont indépendantes ou non. La statistique du $\chi^2$ est donnée par :
    \[
    D_{\chi^2} = n \sum_{j=1}^J \sum_{k=1}^K \frac{(f_{jk} - f_{j\cdot} f_{\cdot k})^2}{f_{j\cdot} f_{\cdot k}}.
    \]
    \item Les coefficients mesurant la dépendance :
    \begin{itemize}
        \item $\phi = \sqrt{\frac{D_{\chi^2}}{n}}$ : intensité de la relation.
        \item $V$ de Cramér : plus précis pour les tableaux non carrés.
    \end{itemize}
    \item Interprétation de $V$ :
    \begin{itemize}
        \item $V < 0.1$ : relation faible ou nulle.
        \item $V > 0.3$ : relation forte.
    \end{itemize}
\end{itemize}

\section*{Régression linéaire}

\subsection*{Nuage de points}
Chaque individu est représenté par un point $(x_i, y_i)$. Le point moyen est donné par $(\bar{x}, \bar{y})$, centre de gravité du nuage.

\subsection*{Méthode des moindres carrés}
\begin{itemize}
    \item L’objectif est de minimiser la somme des carrés des distances verticales entre les points et la droite de régression :
    \[
    a = \frac{\text{Cov}(x, y)}{\sigma_x^2}, \quad b = \bar{y} - a\bar{x}.
    \]
    \item La covariance est calculée par :
    \[
    \text{Cov}(x, y) = m(xy) - \bar{x} \bar{y}.
    \]
\end{itemize}

\subsection*{Coefficient de corrélation linéaire ($r$)}
\[
r = \frac{\text{Cov}(x, y)}{\sigma_x \sigma_y}.
\]
$r = 1$ ou $r = -1$ indique une relation parfaitement linéaire.

\section*{Tests statistiques}

\subsection*{Hypothèses et p-value}
\begin{itemize}
    \item L’hypothèse nulle ($H_0$) est la proposition initiale à tester.
    \item La p-value est la probabilité d’erreur si $H_0$ est rejetée.
    \item On rejette $H_0$ si $p \leq \alpha$ (généralement $\alpha = 0.05$).
\end{itemize}

\subsection*{Test du $\chi^2$}
Le test du $\chi^2$ permet de tester l’indépendance entre deux variables qualitatives.

\subsection*{ANOVA à un Facteur}
L’ANOVA évalue si une variable qualitative (facteur) explique une différence significative entre les moyennes d’une variable quantitative.

\end{document}
